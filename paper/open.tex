\section{The Openness Principle in Lawful Surveillance}
In this section, we review the openness principle put forth by Segal 
{\it et al.}~\cite{sff-foci2014}.  Readers familiar with \cite{sff-foci2014} 
may skip to the next section.

Necessary to any meaningful discussion of ``accountable surveillance'' is an
established foundation of rule of law and democratic processes that subject
the laws to evaluation, debate, and revision.  Bulk surveillance must follow
{\it open processes}, {\it i.e.}, unclassified procedures laid out in public 
laws that all citizens have the right to read, to understand, and to challenge
through the political process.  Processes that are not open, public, and
unclassified in this sense are referred to as {\it secret processes}.  
Although government agencies need not always disclose all of the details of a 
particular investigation, they do need to follow the open processes 
established for all bulk surveillance.


More precisely, Segal {\it et al.}~\cite{sff-foci2014} draw a distinction 
between two classes of communication-system users.  {\it Targeted users} are 
those who are under suspicion and are targets of properly authorized warrants.
All others are {\it untargeted users}; they are the vast majority of all 
users of a general-purpose, mass-communication system.  Segal {\it et 
al.}~\cite{sff-foci2014} posit that the
following {\it Openness Principle} should govern all surveillance activity
in a democratic society:

\begin{enumerate}
\item[I]
Any surveillance or law-enforcement process
that obtains or uses private information
about untargeted users shall be
an open, public, unclassified process.
\item[II]
Any secret surveillance or law-enforcement processes shall use only:
\begin{enumerate}
\item[(a)] public information, and
\item[(b)] private information about targeted users obtained under 
authorized warrants via open surveillance processes.
\end{enumerate}
\end{enumerate}
Segal {\it et al.} interpret this principle as a requirement that an open
``privacy firewall'' be placed between government agencies and citizens'
private information in a mass-communication network.  Processes that move
untargeted users' private information through the firewall must be open
processes.  More generally, the openness principle asserts that, by using
appropriate security technology, government agencies can make their 
data-collection {\it processes} fully public without revealing sensitive
{\it content} of a specific investigation.  For a more detailed explanation,
see Section 2 of \cite{sff-foci2014}.

