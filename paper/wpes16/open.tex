\section{The Openness Principle in Lawful Surveillance}\label{sec-open}
In this section, we review the openness principle put forth by
Bandits~\cite{sff-foci2014}.  Readers familiar with \cite{sff-foci2014} 
may skip to the next section.

Necessary to any meaningful discussion of ``accountable surveillance'' is an
established foundation of rule of law and democratic processes that subject
the laws to evaluation, debate, and revision.  Bulk surveillance must follow
{\it open processes}, {\it i.e.}, unclassified procedures laid out in public 
laws that all citizens have the right to read, to understand, and to challenge
through the political process.  Processes that are not open, public, and
unclassified in this sense are referred to as {\it secret processes}.  
Although government agencies need not always disclose all of the details of a 
particular investigation, they do need to follow the open processes 
established for all bulk surveillance.

More precisely, Bandits~\cite{sff-foci2014} draw a distinction 
between two classes of communication-system users.  {\it Targeted users} are 
those who are under suspicion and are targets of properly authorized warrants.
All others are {\it untargeted users}; they are the vast majority of all 
users of a general-purpose, mass-communication system. Bandits~\cite{sff-foci2014}
posit that the following {\it Openness Principle} should govern all surveillance
activity in a democratic society:
\begin{enumerate}
\item[I]
Any surveillance or law-enforcement process
that obtains or uses private information
about untargeted users shall be
an open, public, unclassified process.
\item[II]
Any secret surveillance or law-enforcement processes shall use only:
\begin{enumerate}
\item[(a)] public information, and
\item[(b)] private information about targeted users obtained under 
authorized warrants via open surveillance processes.
\end{enumerate}
\end{enumerate}
Bandits interpret this principle as a requirement that an open
``privacy firewall'' be placed between government agencies and citizens'
private information in a mass-communication network.  Processes that move
untargeted users' private information through the firewall must be open
processes.  The targeted class contains both {\it known} users, {\it i.e.},
those for whom the government has a name, address, phone number, email address,
or other piece of personally identifying information, and {\it unknown} users.
It is not, as it may seem on the surface, oxymoronic to call a user both
``targeted'' and ``unknown,'' because ambient information may justify the
targeting of an individual without identifying him or her in any standard sense
of ``identify.''  For example, a government agency may obtain a ``John Doe
warrant''~\cite{bieber-penn2002} to investigate users who were present in
locations $L_i$ at times $T_i$, for $1\leq i\leq k$, without being able to 
identify those users, because relevant events occurred at those locations at
those times. Bandits~\cite{sff-foci2014} show how an 
accountable-surveillance protocol can be used to obtain a large set of
{\it encrypted} data about both targeted and untargeted users, feed it into a 
cryptographic protocol that winnows it down to the records of users targeted by
the John Doe warrant, and decrypt {\it only} those records.  Thus, targeted
unknown users can be identified ({\it i.e.}, turned into targeted, known users)
without government agencies' identifying any untargeted users whose 
encrypted records are touched by the surveillance process.

The essence of the openness principle is that, by using
appropriate security technology, government agencies can make their 
data-collection {\it processes} fully public without revealing sensitive
{\it content} of a specific investigation.  For a more detailed explanation,
see \cite[Section 2]{sff-foci2014}.

