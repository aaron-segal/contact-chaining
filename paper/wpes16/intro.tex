\section{Introduction}\label{sec-introduction}
As networked devices become more available, more capable, and more ubiquitous
in everyday life, tension mounts between users' desire to safeguard their 
personal information and government agencies' desire to use that personal 
information in their pursuit of criminals and terrorists.  Many people assert
that we are faced with an unpleasant, stark choice: Citizens can either have
control over their personal information, or they can have
law-enforcement and intelligence agencies with the tools that they need to
keep the country safe. We regard this stark choice as a false dichotomy and 
assert that, by deploying appropriate technology in the context of 
sound policy and the rule of law, we can have both privacy and security.

In this paper, we continue the development of accountable, privacy-preserving 
surveillance that we began in~\cite{sff-foci2014}. We require that government 
surveillance be conducted according to {\it open processes}, {\it i.e.}, 
unclassified procedures laid out in public laws that all citizens have the
right to read, to understand, and to challenge through the politicial process;
procedures that do not have these properties are referred to as {\it secret
processes}.  We distinguish between {\it targeted users}, who are 
under suspicion and the targets of properly authorized warrants, and
{\it untargeted users}; the latter are the vast majority of all 
users in any general-purpose, mass-communication system, and their private
information must be protected from government scrutiny. In \cite{sff-foci2014},
we applied these principles to the problem of {\it set intersection}. 

We now apply them to {\it contact chaining}.
The goal is to use the topology of a {\it communication 
graph} ({\it e.g.}, a phone-call graph, email graph, or social network) 
to identify associates (or ``contacts'') of 
lawfully targeted users~\cite{techdirt}. 
Agencies can then investigate those associates to determine whether
they deserve further attention.
It is useful to consider both direct contacts, 
{\it i.e.}, users who are neighbors in the communication graph, and
extended contacts, {\it i.e.}, users who are at distance $k$ in the 
communication graph, for an appropriate constant $k$.
%In a phone-call graph, if Alice calls Bob, 
%and Bob calls Charlie, then Alice and Bob are direct contacts (as are Bob and 
%Charlie), but Alice and Charlie are extended contacts (more precisely,
%contacts at distance 2).  
Without accountability and security mechanisms to
limit an investigation's scope, contact chaining in a mass-communication
network can sweep in a huge number of untargeted users.
Section~\ref{sec-chaining} presents an accountable contact-chaining
protocol that bounds the scope of the search, uses encryption to protect 
untargeted users, and is computationally efficient,
with time and communication complexity linear in the size of the output.

We posit that the
{\it Openness Principle} put forth in \cite{sff-foci2014} 
should govern all surveillance activity in a democracy.
\begin{compactenum}
\item[I]
Any surveillance or law-enforcement process
that obtains or uses private information
about untargeted users shall be
an open, public, unclassified process.
\item[II]
Any secret surveillance or law-enforcement processes shall use only:
\begin{compactenum}
\item[(a)] public information and
\item[(b)] private information about targeted users obtained under 
authorized warrants via open processes.
\end{compactenum}
\end{compactenum}
This principle can be viewed as a requirement that an open
``privacy firewall'' be placed between government agencies and citizens'
private information in a mass-communication network.  Processes that move
untargeted users' private information through the firewall must be open
processes.  

We briefly present our contact-chaining results in 
Section~\ref{sec-chaining} and our open questions and future directions
in Section~\ref{sec-future}.  More detail on the Openness
Principle, our results on contact chaining and set intersection, related 
work by others, and future directions can be found in 
\cite{sff-TR,sff-foci2014}.
