\section{Open Problems and Future Work}\label{sec-future}
Thus far, we have explored accountable, privacy-preserving protocols for
set intersection and contact chaining. 
Another operation of potential interest is the retrieval of
targeted users' postings on Facebook and other social networks, including 
those that are shared only with a small subset of the target's 
``friends.'' Accountable surveillance of social-network postings may present 
novel protocol-design challenges, because it deals with one-to-many 
communication, while previous work dealt with pairwise communication.

It may be possible to speed up our contact-chaining protocols by using
elliptic-curve cryptography instead of RSA. Our assumption that
all parties are honest-but-curious might be weakened, {\it e.g.}, by using 
zero-know\-ledge techniques to obtain versions
of our protocols that are secure against
a rogue agent's maliciously modifying telecom-supplied data in order
to falsely incriminate a victim. One may wish to
generalize the differential-privacy approach of Kearns {\it et 
al.}~\cite{krwy-pnas16} to handle to indirect contacts as well as
direct contacts.

Finally, the Openness Principle of \cite{sff-foci2014} is but one step
toward a full understanding of how democratic processes and the rule of law
can be carried into the digital world.  Further investigation, much of it
interdisciplinary, is needed.

%
