\section{Open Problems and Future Work}\label{sec-future}
Segal {\it et al.} noted in \cite[Section 6.1]{sff-foci2014} that
set intersection is but one type of computation that can be of use
to law-enforcement and intelligence agencies.  They observed that it 
would be interesting to identify other such computational problems 
and to devise accountable, privacy-preserving protocols to solve them.  
The work in this paper on contact chaining represents progress in that
direction.  

Another problem of potential interest is the retrieval of
targeted users' postings on Facebook and other social networks, including 
those that are shared only with a small subset of the targeted user's 
``friends.'' Accountable surveillance of social-network postings may present 
novel protocol-design challenges, because it deals with one-to-many 
communication, whereas previous work in the area dealt with pairwise 
communication.

For contact chaining, it may be possible to speed up our protocols by using
elliptic-curve cryptography instead of RSA.  It may also be interesting to 
generalize the differential-privacy approach of Kearns {\it et 
al.}~\cite{krwy-pnas16} so that it applies to indirect contacts as well as
direct contacts. 

Finally, the Openness Principle put forth in \cite{sff-foci2014} is but one step
toward a full understanding of how democratic processes and the rule of law
can be carried into the digital world.  Further investigation, much of it
interdisciplinary, is needed.

