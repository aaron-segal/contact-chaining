\section{Related Work}\label{sec-related}
Privacy-preserving computation has been studied extensively. An overview
of the approaches taken by the cryptographic-research community is provided by
Perry {\it et al.}~\cite{pgfw-scn2014}; an overview of of the data-mining
approach is provided by Aggarwal and Yu~\cite{ay2008}.  

Kamara~\cite{kamara}
and Kroll {\it et al.}~\cite{kroll} used cryptographic protocols to 
achieve privacy and accountability in the surveillance of known targets.
Segal {\it et al.}~\cite{sff-foci2014} formulated the openness principle that
we have followed and were the first to design privacy-preserving protocols for
the surveillance of unknown targets.  

Kearns {\it et al.}~\cite{krwy-pnas16} present efficient graph-search 
algorithms that distinguish targeted users from untargeted users; for each 
untargeted user $u$, the set of direct contacts of $u$ remains private.
Unlike our privacy-preserving contact-chaining algorithms, which rely
on cryptographic techniques, their graph-search algorithms rely on 
differential-privacy techniques.
