\section{Introduction}
As networked devices become more available, more capable, and more ubiquitous
in everyday life, tension mounts between users' desire to safeguard their 
personal information and government agencies' desire to use that personal 
information in their pursuit of criminals and terrorists.  For example, 
both the heated (and still unresolved) discussion about the Snowden 
revelations that started in 2013 and the ongoing dispute between Apple and 
the US Justice Department are understood by many people
as examples of an unpleasant, stark choice: Citizens can either have
secure devices and control over their personal information, or they can have
law-enforcement and intelligence agencies with the tools that they need to
keep the country safe. We regard this stark choice as a false dichotomy and 
assert that, by deploying appropriate security technology in the context of 
sound policy and the rule of law, we can have both user privacy and effective 
law enforcement and intelligence.

Continuing the approach taken by Segal {\it et al.}~\cite{sff-foci14}, we
seek to design and implement protocols for {\it accountable surveillance}.

At first blush, it may seem that a symposium on ``privacy-enhancing 
technologies'' is an odd place for results about ``accountable surveillance.''
No doubt some in the PETS community wish to prevent government agencies 
(as well as large corporations and other powerful entities) from conducting 
{\it any} surveillance whatsoever.  As explained in \cite{sff-foci2014}, a 
global communication system entirely free of surveillance may be appealing in
the abstract, but it is not a very useful goal in practice.  Law enforcement
and intelligence agencies have always been and will continue to be active 
on the Internet and in all national- and global-scale communication systems.
The challenge for the technical community is to build systems that enable
government agencies to collect relevant data that they are legally authorized
to collect, to be held accountable to the citizens they serve, and to respect 
privacy of innocent users.

