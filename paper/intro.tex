\section{Introduction}
As networked devices become more available, more capable, and more ubiquitous
in everyday life, tension mounts between users' desire to safeguard their 
personal information and government agencies' desire to use that personal 
information in their pursuit of criminals and terrorists.  For example, 
both the heated (and still unresolved) discussion about the Snowden 
revelations that started in 2013 and the ongoing dispute between Apple and 
the US Justice Department are understood by many people
as examples of an unpleasant, stark choice: Citizens can either have
secure devices and control over their personal information, or they can have
law-enforcement and intelligence agencies with the tools that they need to
keep the country safe. We regard this stark choice as a false dichotomy and 
assert that, by deploying appropriate security technology in the context of 
sound policy and the rule of law, we can have both user privacy and effective 
law enforcement and intelligence.

Continuing the approach taken by Segal {\it et al.}~\cite{sff-foci2014}, we
seek to design and implement protocols for {\it accountable surveillance}.
We require that government surveillance be conducted according to {\it open
processes}, as defined in Section~\ref{sec-open} below, and that the privacy
of {\it untargeted users} be protected.  We consider the surveillance goals
of {\it set intersection} and {\it contact chaining} and show that both can 
be achieved in a privacy-preserving, accountable fashion.

The utility of set-intersection protocols was demonstrated in the 
High Country Bandits case~\cite{anderson13cell}.
After obtaining cell-tower dumps -- sets of about 150,000 total users
whose cell phones had been in the vicinity of three banks at the times
that those banks were robbed -- the FBI intersected the sets and discovered 
that a single phone had been used at all of the relevant times in all of the 
relevant places.  They arrested the owner
of that phone and were able to prove that he was one of the robbers.
Although this FBI dragnet was effective in catching robbers, it also swept
in the cell-phone numbers of approximately 149,999 innocent bystanders.
Segal {\it et al.}~\cite{sff-foci2014} provide an accountable protocol for
set intersection that preserves the privacy of innocent bystanders.  Their
rudimentary implementation requires just under two hours on a test instance
with 150,000 total users.  In Section~\ref{sec-intersection} below, we provide
a more careful implementation that is faster by a factor of 10; in particular,
it runs for approximately 10.5 minutes on the test instance of 150,000 users.

In Section~\ref{sec-chaining}, we turn our attention to 
accountable-surveillance protocols for contact chaining.

At first blush, it may seem that a symposium on ``privacy-enhancing 
technologies'' is an odd place for results about ``accountable surveillance.''
No doubt some in the PETS community wish to prevent government agencies 
(as well as large corporations and other powerful entities) from conducting 
{\it any} surveillance whatsoever.  As explained in \cite{sff-foci2014}, a 
global communication system entirely free of surveillance may be appealing in
the abstract, but it is not a very useful goal in practice.  Law enforcement
and intelligence agencies have always been and will continue to be active 
on the Internet and in all national- and global-scale communication systems.
The challenge for the technical community is to build systems that enable
government agencies to collect relevant data that they are legally authorized
to collect, to be held accountable to the citizens they serve, and to respect 
privacy of innocent users.

